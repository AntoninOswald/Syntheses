\documentclass[fr]{../../../../../../eplexam}

\hypertitle{Calculabilité}{6}{INGI}{1123}{2019}{Juin}{Majeure}
{Antoine Grollinger \and Donatien Schmitz \and François De Keersmaeker \and François Duchêne \and Julien Mans}
{Yves Deville}

\section{Machine de Turing}
\textit{4 pts}
Soit la machine de Turing suivante avec $\Sigma = \{0,1\}$
\begin{tabular}{|l|l||l|l|r|}
	\hline 
	État & Symbole & État & Symbole & Action \\ 
	\hline 
	S0 & 0 & S1 & 0 & D \\ 
	\hline 
	S0 & 1 & S1 & 1 & D \\ 
	\hline 
	S0 & B & Stop & B & G \\ 
	\hline 
	S1 & 0 & S0 & 1 & D \\ 
	\hline 
	S1 & 1 & S0 & 0 & D \\ 
	\hline 
	S1 & B & Stop & B & G \\ 
	\hline 
\end{tabular} 
\begin{description}
	\item[Machine déterministe ?] Vrai/Faux
	\item[Machine a 2 etats ?] Vrai/Faux
	\item[Sortie pour entrée 110011 ?]
\end{description}
\begin{solution}
\begin{description}
	\item[Machine déterministe ?] Vrai, une seule exécution est possible pour une entrée donnée
	\item[Machine a 2 etats ?] Faux, 3 états: S0, S1 et Stop
	\item[Sortie pour entrée 110011 ?] 100110, cette entrée est acceptée si Stop est acceptant
\end{description}

\end{solution}

\section{Questions vrai/faux avec justification}
\textit{7 pts}
\begin{enumerate}
	\item L’ensemble des fonctions de $\N$ vers $\{0,1\}$ est non énumérable
	\item Il n’existe pas de langage tel que \emph{halt} est calculable dans ce langage
	\item Un sous ensemble infini d’un ensemble récursivement énumérable est récursivement énumérable 
	\item Le théorème du Point Fixe est une conséquence du théorème de Rice
	\item Une grammaire hors contexte ne permet que de construire des langages récursifs
	\item Un formalisme \(D\) qui a un interpréteur calculable possède la propriété U
	\item Si $A \in NP$ et $A \le_{p} SAT$ alors $A$ est NP-Complet.
\end{enumerate}

\begin{solution}
\begin{enumerate}
	\item Vrai, supposons que c'est le cas et construisons le tableau :
\begin{center}
\begin{tabular}{| c | c c c c |}
\hline
$f_0$ & $f_0(0)$ & $f_0(1)$ & $f_0(2)$ & $\cdots$\\
\hline
$f_1$ & $f_1(0)$ & $f_1(1)$ & $f_1(2)$ & $\cdots$\\
\hline
$\vdots$&$\vdots$&$\vdots$&$\vdots$&$\ddots$\\
\hline
\end{tabular}
\end{center}
On prend alors la diagonale $d := (f_k(k))_{k\in\mathbb{N}}$ que l'on remplace par $d' := (\neg f_k(k))_{k\in\mathbb{N}}$.\\
Par notre hypothèse, $d'$ est également dans le tableau, on a donc que $\exists p \in \mathbb{N} : d' = f_p$ soit 
\begin{alignat*}{6}
\neg 	&f_0(0), \neg 	&f_1(1),		&\cdots, \neg 	&f_p(p), &\cdots\\
=		&f_p(0), 		&f_1(1),		&\cdots, 		&f_p(p), &\cdots
\end{alignat*}
$\Rightarrow \neg f_p(p) = f_p(p)$, ce qui est une contradiction, on a donc prouvé que l'ensemble des fonctions de $\mathbb{N}$ dans $\{0,1\}$ est non énumérable.
	\item Faux, BLOOP est un exemple (tous les programmes BLOOP se terminent donc \emph{halt} retourne simplement 1). Il n'existe cependant pas de langage tel que à la fois l'interpréteur de celui-ci et halt sont calculables dans ce langage.
	\item 
	\item Faux, c'est l'inverse (Syllabus Remarque 3.64).
	\item Vrai, elle correspond à une grammaire de type 2 dans la hiérarchie de Chomsky.
	\item Faux, il doit aussi avoir la propriété CA ou encore les propriétés CD et S.
	\item Faux, il faut que SAT $\leq_p$ A, pas l'inverse.
\end{enumerate}
\end{solution}

\section{Énoncer précisément le théorème de Rice, donner sa signification et ses conséquences}
\textit{4 pts}

\begin{solution}
Si A est récursif et $A \neq \emptyset, A \neq \mathbb{N}$, alors $\exists i \in A$ et $\exists j \in \mathbb{N} \backslash A$ tel que $\varphi_{i}=\varphi_{j}$.\\
Si $A$ est récursif, qu’il n’est pas vide et qu’il ne contient pas l’ensemble des programmes existant, alors
il existe un programme dans $A$ et un programme dans $\bar{A}$ tel qu’ils calculent la même fonction.\\
Sa concéquence la plus utile est sa contraposée: Si $\forall i \in A$ et $\forall j \in \bar{A}$ on a $\varphi_{i} \neq \varphi_{j}$, alors $A$ est non-récursif ou $A=\emptyset$ ou $A=\mathbb{N}$. \\Comme il est souvent simple de garantir $A=\emptyset$ et $A=\mathbb{N}$, on a un bon moyen de prouver que $A$ est non-récursif.
\end{solution}

\section[Réduction polynomiale et SAT appartenant à P]{Définir la réduction polynomiale et prouver que si $SAT \in P$ alors $P=NP$}
\textit{4 pts}

\begin{solution}
Un ensemble $A$ est polynomialement réductible à un ensemble $B$, $A \leq_{p} B$ s'il existe une fonction totale calculable $f$ de complexité temporelle polynomiale telle que
$$a \in A \Leftrightarrow f(a) \in B$$
Comme $A\leq_pSAT ~~ \forall A\in NP$ (SAT est $NP$-Complet), on pourrait alors réduire n'importe quel problème dans $NP$ à SAT en complexité plynomiale puis résoudre SAT en temps polynomial également, ce qui entrainerait $P=NP$.
\end{solution}

\end{document}
